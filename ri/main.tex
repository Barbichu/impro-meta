\documentclass[a4paper,french,10pt]{article}
\usepackage[french]{babel} 
\usepackage[utf8]{inputenc}
\usepackage[T1]{fontenc} %\usepackage{isolatin1} \usepackage{t1enc}
\usepackage{pslatex} % Pour une meilleure police avec le PDF
\usepackage{ae,aecompl} 
\usepackage{marvosym} %pour l'¤
\usepackage{url} 
\usepackage{color} 
\usepackage{shadow}
\usepackage{fancyhdr} 
\usepackage{fancybox}

\usepackage[colorlinks=true,urlcolor=black,pdfauthor={Association
  META},pdftitle={Statuts de l'assocation
  META},pdfsubject={Association META : Improvisation Théatrâle, Basée
  à Paris},pdfkeywords={META, improvisation, impro, théatre,
  spectacle, paris, association loi 1901, statuts }]{hyperref}


\renewcommand{\headrulewidth}{0.4pt} %Trait haut
\renewcommand{\footrulewidth}{0pt} %Trait bas

\newcommand{\meta}{\textbf{META}}

\title{Règlement intérieur de l'association \meta{}}

\begin{document}

\date{}

\maketitle 
\pagestyle{fancy} 

\lhead{\textit{Règlement Intérieur de l'association \meta{}}}

%Haut gauche \chead{} 
%Haut centre \rhead{\thepage} 
%Haut droite
\rhead{\thepage} 
%Bas gauche 
\cfoot{} 
%Bas centre \rfoot{} 
%Bas droite \lfoot{}


\newcounter{article}
\setcounter{article}{1}

Le règlement intérieur a pour objet de préciser les statuts de l'association \meta, dont le siège est au 40 Rue Montgallet 75012 Paris , et dont l'objet est :
\begin{itemize}
\item la pratique, la promotion et la diffusion de l'improvisation théâtrale au travers d'entraînements et de spectacles.
\item la création et la diffusion de formats originaux.
\item la réflexion autour de l'improvisation.
\end{itemize}

Le présent règlement intérieur est transmis à l'ensemble des membres ainsi qu'à chaque nouvel adhérent.

\section{Membres}
\subsection{Cotisation}

Montant des cotisations :

\begin{itemize}
\item Les membres d'honneur ne paient pas de cotisation sauf s'ils décident de s'en acquitter de leur propre volonté.
\item Les membres bienfaiteurs doivent s'acquitter d'une cotisation libre supérieure ou égale à 15 euros.
\item Les membres adhérents doivent s'acquitter d'une cotisation annuelle de 15 euros.
\end{itemize}


Sur demande de l'Assemblée Générale ou du trésorier, le montant de la cotisation peut-être modifié. Il ne pourra être inférieur à 5 euros et ne pourra baisser que si le budget dégage suffisemment de bénéfices sur l'exercice précédent pour couvrir les dépenses prévisionnelles de l'exercice suivant. Le montant sera fixé selon la procédure suivante :
\begin{itemize}
\item Le trésorier présente le bilan financier de l'association pour l'année écoulée.
\item Il présentera ensuite un projet prévisionnel pour l'année à suivre, indiquant les dépenses et recettes prévues.
\item le trésorier inscrit un montant de réserve à minima sur un bulletin secret permettant de couvrir les frais prévisionnels.
\item L'Assemblée Générale débattra (sans intervention du trésorier) afin de proposer un montant de cotisation approuvé par la majorité de l'Assemblée Générale.
\item Le montant retenu est le montant le plus élevé du montant proposé par l'Assemblée Générale et du montant de réserve.
\end{itemize}

\subsubsection{Première adhésion}

Le versement de la cotisation annuelle doit être établie par chèque ou argent liquide à l'ordre de l'association et effectué dans les deux mois suivants la lettre d'admission du bureau. La paiement de la cotisation donne la qualité de membre jusqu'au 31 Septembre prochain.

Les adhésions au mois de Juillet et Août sont gratuites.

\subsubsection{Ré-adhésion}

Le versement de la cotisation annuelle pour les membres de l'association doit être établi par chèque ou argent liquide à l'ordre de l'association et effectué avant le 31 Octobre de l'année en cours.

\subsubsection{Remboursements}

Toute cotisation versée à l'association est définitivement acquise. Un remboursement de cotisation en cours d'année ne peut être exigé en cas de démission, d'exclusion ou de décès d'un membre.

\subsection{Admission de nouveaux membres}

L'association \meta a vocation à accueillir de nouveaux membres sur décision du bureau. Ceux-ci devront respecter la procédure d'admission suivante : dépôt d'une demande écrite auprès du bureau, et passage d'un entraînement d'essai. Le bureau fournira une réponse écrite dans les 15 jours suivant l'entraînement d'essai informant le demandeur de sa décision appuyé d'une justification.

\subsection{Exclusion}
Conformément à l'article 8 des statuts de l'association, ne peuyvent induire une procédure d'exclusion seuls les cas de :
\begin{itemize}
\item non-participation à l'association pendant un délai de 3 ans,
\item défaut de paiement de la cotisation annuelle,
\item injure ou menace d'un autre membre de l'association,
\item comportement irresponsable entraînant la mise en danger d'autruis.
\end{itemize}

Celle-ci doit être prononcée par l'Assemblée Générale, après avoir entendu le membre contre lequel la procédure d'exclusion est engagée.

\subsection{Démission - Décès - Disparition}
Le membre démissionnaire devra adresser sous lettre sa décision au bureau.
Aucune restitution de cotisation n'est due au membre démissionnaire.

En cas de décès, la qualité de membre disparaît avec la personne.

\section{Organisation et fonctionnement de l'association}
\subsection{Fonctionnement du bureau}

\subsection{Assemblée générale ordinaire}
Conformément à l'article 12 des statuts, l’Assemblée Générale ordinaire se réunit au moins une fois par an et comprend tous les membres de l’association à jour de leur cotisation.

Quinze jours au moins avant la date fixée, les membres de l’association sont convoqués par écrit et l’ordre du jour est inscrit sur les convocations.

Une Assemblée Générale doit être tenue la dernière semaine de Septembre. L'Assemblée Générale statue sur le renouvellement du bureau.

Les modalités de renouvellement du bureau sont les suivantes : sont éligibles tous les membres adhérents à jour de leur cotisation. L'Assemblée Générale se pronoce pour chaque poste séparément.

\subsection{Assemblée Générale Extraordinaire}
\label{sec:age}
Conformément à l'article 13 des statuts, l'Assemblée Générale Extraordinaire peut être réunie sur demande d'au moins la moitié des membres inscrits ou de sa propre initiative, le bureau.
\subsection{Modalités communes aux Assemblée Générale Ordinaire et Extraordinaires}
\label{sec:ag}
L'ordre du jour est établi par le bureau. % ou le président ?
% Seuls les points à l'ordre du jour peuvent être votés

Aucun quorum n'est requis. Le vote est effectué à bulletin secret ou à
main levée, suivant la nature de la décision : le choix est laissé à
l'initiative du président, excepté l'élection des membres du Bureau
pour laquelle le scrutin secret est requis.
% enlever le "excepté [...]" ?
% O : Non je pense pas

Les décisions de l’assemblée sont prises à la majorité relative des
membres présents et représentés. En cas de litige, le vote du
président compte double.

Seuls les membres de l’association à jour de leur cotisation
bénéficient du droit de vote et sont éligibles.
\subsection{Transparence des comptes}
\label{sec:transp-des-compt}
% cf statuts improdisaque article 11b

\subsection{Fonctionnement de la Troupe}
\label{sec:fonctionnement-troupe}
La Troupe est un collège de membres adhérents chargé d'assuré la représentation des spectacles. La Troupe est composée d'au moins 6 (six) membres adhérents. Tout nouveau recrutement d'un membre dans la Troupe se fera par cooptation à l'unanimité des membres de la Troupe. La troupe se réunit en collège -- pouvant prendre la forme de réunions privées -- au moins une fois tous les 2 (deux) mois.

La Troupe propose au Bureau un Directeur Artistique, choisi parmi les membres de l'association. Le Directeur Artistique est chargé de rédiger et de mener à bien le projet artistique de la Troupe en tenant compte des orientations décidées par le Bureau. Il est aussi responsable de l'organisation de la réflexion menant à l'aboutissement d'au moins 1 (un) format innovant de spectacle. Le Bureau peut refuser la proposition de la Troupe. 

La qualité de membre de la Troupe peut se perdre par les moyens suivants :

\begin{enumerate}
	\item la démission remise au Directeur Artistique ;
	\item l'absence pendant plus de 3 (trois) mois aux entraînements spectacles ou plus de 5 (cinq) absences successives aux entraînements spectacles sans justification ;
% à voir au sujet justification...
	\item le vote à la majorité de la totalité de ses membres lors d'un collège ;
	\item la perte de qualité de membre de l'association

\end{enumerate}



\section{Organisation des activités artistiques}
\subsection{Entraînements}
L'association organise deux types d'entraînements : entraînements réguliers et entraînement spectacles.

\subsubsection{Entraînements réguliers}
L'association organise au moins une fois par mois un entraînement. Un animateur d'entraînement désigné par le Directeur Artistique sera responsable du contenu de l'entraînement. Tous les membres adhérents et d'honneur peuvent y participer. Tous les membres bienfaiteurs peuvent y assister.

\subsubsection{Entraînements spectacles}
Sur décision du Directeur Artistique, des entraînements réservés à la Troupe peuvent être organisés. Des membres adhérents ou membres d'honneurs peuvent être invités par la Troupe à y participer. Des membres peuvent être invités à y assister.

\subsection{Spectacles}
\label{sec:spectacles}


\section{Disposition diverses}
\subsection{Règlement intérieur}
Conformément à l'article 19 des statuts de l'association, le règlement intérieur est établis par le bureau.

Il peut être modifié sur demande de l'Assemblée Générale ou du bureau, suivant la procédure suivante :
Le bureau rédige une proposition soumise à l'approbation de l'Assemblée Générale ou par vote par courrier dans un délai de 15 jours.
Le nouveau règlement intérieur est alors adressé à tous les membres par courrier.
\end{document}
