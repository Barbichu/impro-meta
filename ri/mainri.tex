\setcounter{article}{1}
\section{Membres}

\article{Admission de nouveaux membres}
\label{sec:admiss-de-nouv}

Conformément à l'\artrefst{sec:admission}, l'association \meta{} a vocation à accueillir de nouveaux membres sur décision du \bureau{}. Ceux-ci devront respecter la procédure d'admission suivante : dépôt d'une demande écrite auprès du \bureau{}, et passage d'un entraînement d'essai. Le \bureau{} fournira une réponse écrite dans les quinze (15) jours suivant l'entraînement d'essai informant le demandeur de sa décision appuyée d'une justification. Les nouveaux membres se verront remettre une copie du \RI{}, et s'acquitteront d'une cotisation précisée dans l'article suivant.

\article{Cotisation}
\label{sec:cotisation}
Montant des cotisations :

\begin{itemize}
\item Les membres d'honneur ne paient pas de cotisation sauf s'ils décident de s'en acquitter de leur propre volonté.
\item Les membres bienfaiteurs doivent s'acquitter d'une cotisation libre supérieure ou égale à cinq euros (5 \euro{}).
\item Les membres adhérents doivent s'acquitter d'une cotisation annuelle de quinze (15 \euro{}).
\end{itemize}


Sur demande de l'\AG{} ou du trésorier, le montant de la cotisation peut-être modifié. Conformément à l'\artrefst{sec:admission}, il ne peut-être inférieur à cinq euros (5 \euro{}) et ne pourra baisser que si le budget dégage suffisamment de bénéfices sur l'exercice précédent pour couvrir les dépenses prévisionnelles de l'exercice suivant. Le montant sera fixé selon la procédure suivante :
\begin{itemize}
\item le trésorier présente le bilan financier de l'association pour l'année écoulée ;
\item il présente ensuite un projet prévisionnel pour l'année à suivre, indiquant les dépenses et recettes prévues ;
\item puis inscrit un montant de réserve à minima sur un bulletin secret permettant de couvrir les frais prévisionnels ;
\item l'\AG{} débat (sans intervention du trésorier) afin de proposer un montant de cotisation approuvé par la majorité de l'\AG{} ;
\item le montant retenu est le montant le plus élevé du montant proposé par l'\AG{} et du montant de réserve.
\end{itemize}

\subsubsection*{Première adhésion}

Le versement de la cotisation annuelle doit être établie par chèque à l'ordre de l'association, par liquide, ou par virement bancaire, et effectué dans les deux mois suivants la lettre d'admission du \bureau{}. La paiement de la cotisation donne la qualité de membre jusqu'au 31 septembre suivant.

La cotisation des adhésions du mois de juillet et août sont réduites à un euro (1  \euro{}) payables avant le 31 octobre suivant.

\subsubsection*{Ré-adhésion}

Le versement de la cotisation annuelle pour les membres de l'association doit être établi par chèque à l'ordre de l'association, par liquide, ou par virement bancaire, et effectué avant le 31 octobre de l'année en cours.

\subsubsection*{Droits de l'adhérent}
Les adhérents sont informés que l'association met en œuvre un traitement automatisé des informations nominatives les concernant.
Ce fichier est à l'usage exclusif de l'association ; il présente un caractère obligatoire. L'association s'engage à ne pas publier ces données nominatives sur Internet sans le consentement de chaque adhérent.
Les informations recueillies sont nécessaires pour à l'adhésion. Elles font l'objet d'un traitement informatique et sont destinées au secrétariat de l'association. Elles peuvent donner lieu à l'exercice du droit d'accès et de rectification selon les dispositions de la loi du 6 janvier 1978. Pour exercer ce droit et obtenir communication des informations le concernant, l'adhérent s'adressera au secrétaire ou à un membre de l'association.

\subsubsection*{Remboursements}

Toute cotisation versée à l'association est définitivement acquise. Un remboursement de cotisation en cours d'année ne peut être exigé en cas de démission, d'exclusion ou de décès d'un membre.

\article{Exclusion}
\label{sec:exclusion}

Conformément à l'\artrefst{sec:perte} de l'association, seul les cas suivants  peuvent induire une procédure d'exclusion :
\begin{itemize}
\item non-participation à l'association pendant un délai de trois (3) ans,
\item défaut de paiement de la cotisation annuelle,
\item injure à l'égard de ou menace sur un autre membre de l'association,
\item comportement irresponsable entraînant la mise en danger d'autrui.
\end{itemize}

L'exclusion doit être prononcée par l'\AG{}, après avoir entendu le membre contre lequel la procédure d'exclusion est engagée.

\article{Démission - Décès - Disparition}
\label{sec:demiss-deces-disp}
Le membre démissionnaire devra adresser sous lettre sa décision au \bureau{}.
Aucune restitution de cotisation n'est due au membre démissionnaire.

En cas de décès, la qualité de membre disparaît avec la personne.

\section{Organisation et fonctionnement de l'association}

\article{Communication des adhérents}
\label{sec:comm-des-adher}

Conformément à l'\artrefst{sec:moyen-de-corresp}, les adhérents peuvent envoyer leurs communications écrites par lettre signées ou également par :
\begin{itemize}
\item e-mail doté d'une clé gpg signée par l'un des membres du \bureau{}
\item formulaire internet sécurisé sur le site de l'association
%\item jabber ?
\end{itemize}


\article{Fonctionnement du \bureau{}}
\label{sec:fonct-du-bure}

\subsubsection*{Composition - Désignation - Démission}
\label{sec:comp-design}
Il est composé d'exactement trois membres dont :
\begin{itemize}
\item un(e) président(e)
\item un(e) trésorier(e)
\item un(e) secrétaire
\end{itemize}

Le président, le trésorier et le secrétaire sont élus au sein de l'\AG{} par un scrutin nominal à un tour pour chaque fonction. On commence par l'élection du président, puis du trésorier et enfin du secrétaire.  Les candidats sont élu à la majorité des membres présents lors de l'\AG{}.

Les fonctions de président et de trésorier sont interdites aux mineurs. Les candidats doivent déclarer formellement leur volonté d'être candidats, par écrit le jour de l'\AG{} ou bien par e-mail à l'un des membres du \bureau{} sortant, au moins une semaine avant l'\AG{}.


Il est possible d'être représenté à une \AG{} en donnant, par écrit, une procuration nominale, une même personne ne pouvant détenir plus
d'une procuration (cf \artref{sec:ag}).

Si l'un des membre du \bureau{} démissionne, perd son statut de membre adhérent ou se trouve dans l'incapacité d'assurer ses fonctions depuis au moins trente (30) jours et pour une durée indéterminée, il est alors remplacé temporairement par un membre adhérent coopté par le \bureau{}. Une \AGE{} doit être organisée dans les trente (30) jours suivant le constat de ce cas afin de pourvoir le poste lors d'un vote à bulletin secret.

\subsubsection*{Fonctions}
\label{sec:fonctions}

Les membres du \bureau{} prennent en charge les trois fonctions de
l'association.  Ils disposent à cet effet des pleins pouvoirs,
notamment pour engager juridiquement l'association et la représenter
en justice, dans le respect des dispositions statutaires.

L'association donne tous les moyens aux dirigeants pour mener à bien
leurs tâches, y compris le recours à la sous-traitance ou la collecte
d'avis d'experts.

Les membres du \bureau{} disposent des pleins pouvoirs pour conduire les
activités de l'association et engager à cet effet les différentes
ressources de l'association.

Les membres du \bureau{} veillent à la tenue des projets artistiques, à
l'entente des membres, au respect des grands équilibres financiers
et à la sécurité de toutes les parties prenantes. 


Le président ou un membre du \bureau{} assure la direction opérationnelle de
l'association. Il dispose à cet effet de tout pouvoir pour notamment :
\begin{itemize}
\item organiser la pratique des activités, en mobilisant les ressources de l'association, sécuriser les conditions d'exercice (notamment en interrompant les activités dès lors que les conditions de sécurité ne seraient pas réunies).
\item sélectionner les locaux, en assurer le pilotage, organiser l'engagement des bénévoles.
\end{itemize}

Le \bureau{} représente l'association tant à l'égard des pouvoirs publics qu'auprès des partenaires privés.

Le \bureau{} négocie et conclut tous les engagements de l'association et d'une manière générale, agit au nom de l'organisme en toutes circonstances, sous réserve du respect des \statuts{} et des décisions souveraines de l'\AG{}.

Le trésorier ou un membre du \bureau{} veille au respect des grands équilibres financiers de l'association, en maîtrisant les dépenses, assurant un flux de recettes internes et externes suffisant et en fixant des tarifs équilibrés. Ils assurent ou font assurer par les ressources bénévoles, salariées ou externes de l'association, les tâches suivantes :
\begin{itemize}
\item Le suivi des dépenses et des comptes bancaires ;
\item La préparation et le suivi du budget ;
\item La transparence du fonctionnement financier envers l'\AG{} ;
\item Les demandes de subventions ;
\item L'établissement de la comptabilité.
\end{itemize}

Le secrétaire ou un membre du \bureau{} veille au respect de la réglementation tant interne qu'externe. Ils assurent ou font assurer par les ressources bénévoles, salariées ou externes de l'association, les tâches suivantes :
\begin{itemize}
\item La convocation et le bon déroulement de l'\AG{} (convocation, comptes rendus ) ;
\item La bonne circulation des informations à destination des adhérents ;
\item L'archivage de tous les documents juridiques et comptables de l'association ;
\item Les déclarations en préfecture (création, certaines modifications statutaires, changement de dirigeants, acquisitions, dissolution) ;
\item Les publications au journal officiel ;
\item Le dépôt des comptes de résultat, bilan, rapport d'activité et conventions en préfecture dès lors que le financement par les autorités administratives dépasse 153 000 \euro{} (L. du 12 avril 2000, D. du 6 juin 2001) ;
\item Dans les communes de plus de 3 500 habitants, le dépôt en mairie d'un bilan certifié conforme si l'association reçoit de la commune une subvention supérieure à 76 300 \euro{} ou représentant plus de 50\% de son budget.
\end{itemize}

\subsubsection*{Réunion - décisions - votes}

Les modalités de fonctionnement du \bureau{} sont les suivantes : il
doit se réunir au moins une fois par mois. Les décisions sont
prises à l'unanimité des membres du \bureau{}. Sur toutes les questions,
ils recherchent un consensus, chacun d'entre eux pouvant s'opposer à
une décision.

\article{Assemblée générale ordinaire}
\label{sec:assembl-gener-ordin}
Conformément à l'\artrefst{sec:ago}, l'\AG{} ordinaire se réunit au moins une fois par an et comprend tous les membres de l'association à jour de leur cotisation.

Une \AG{} doit être tenue dans le courant du mois de Septembre de chaque année. L'\AG{} statue sur le renouvellement du \bureau{}.

Les modalités de renouvellement du \bureau{} sont les suivantes : sont éligibles tous les membres adhérents à jour de leur cotisation. L'\AG{} se prononce pour chaque poste séparément.

\article{Assemblée Générale Extraordinaire}
\label{sec:age}
Conformément à l'\artrefst{sec:age}, l'\AGE{} peut être réunie sur demande d'au moins la moitié des membres inscrits ou à l'initiative du \bureau{}.

\article{Modalités communes aux Assemblée Générale Ordinaire et Extraordinaires}
\label{sec:ag}

\subsubsection*{Ordre du jour et convocation}

L'ordre du jour est établi par le \bureau{}.

Quinze (15) jours au moins avant la date fixée, les membres de
l'association sont convoqués par courrier ou e-mail et l'ordre du jour est inscrit
sur les convocations.

% mouais ?

\subsubsection*{Procuration}

Il est possible d'être représenté à une \AG{} en donnant,
par écrit au \bureau{} (ou le cas échéant le \bureau{} sortant), une procuration nominale, une même personne ne pouvant détenir plus
d'une procuration.

\subsubsection*{Quorum et vote}

Aucun quorum n'est requis. Le vote est effectué à bulletin secret ou à
main levée, suivant la nature de la décision : le choix est laissé à
l'initiative du président, excepté l'élection des membres du \bureau{}
pour laquelle le scrutin secret est requis.


\subsubsection*{Décisions}

Seules les questions inscrites à l'ordre du jour peuvent être
valablement évoquées en assemblée.

Les décisions de l'assemblée sont prises à la majorité relative des
membres présents et représentés. En cas de litige, le vote du
président compte double.

Seuls les membres de l'association à jour de leur cotisation
bénéficient du droit de vote et sont éligibles.

\article{Transparence des comptes}
\label{sec:transp-des-compt}
% cf statuts improdisaque article 11b
Les comptes de l'association sont consultables sur le site internet de l'association. Ils seront protégés par un mot de passe. Les mises à jour seront effectuées au moins une fois par an, avant l'\AGO{}. Les pièces seront conservées sur le site pendant une durée minimale de 1 an. Au delà, elles pourront être effacées.

Le mot de passe sera communiqué aux membres qui en feront la demande.

\article{Fonctionnement de la Troupe}
\label{sec:fonctionnement-troupe}
\troupe{} est un collège de membres adhérents chargé d'assurer la représentation des spectacles. Le recrutement d'un membre dans \troupe{} se fera par cooptation à l'unanimité des membres de \troupe{}.

 \troupe{} se réunit en collège -- pouvant prendre la forme de réunions privées -- au moins une fois tous les deux (2) mois.

Suite à l'\AGO{} et aux orientations artistiques proposées par le \bureau{}, \troupe{} propose au \bureau{} un \DA{}, choisi parmi les membres de l'association. Le \DA{} est chargé de rédiger et de mener à bien le \PA{} de \troupe{} en tenant compte des orientations décidées par le \bureau{}. Il est aussi responsable de l'organisation de la réflexion menant à l'aboutissement d'au moins un (1) format innovant de spectacle. Le \bureau{} peut refuser la proposition de \troupe{}. 

Le \DA{} assistera à l'ensemble des réunions et entraînements de \troupe{}. Il participera à toutes les décisions prises par \troupe{}, y compris les admissions et exclusions de membres de \troupe{}.

La qualité de membre de \troupe{} peut se perdre par les moyens suivants :

\begin{enumerate}
	\item la démission remise au \DA{} ;
	\item l'absence pendant plus de trois  (3) mois aux entraînements spectacles ou plus de cinq (5) absences successives aux entraînements spectacles sans justification ;
% à voir au sujet justification...
	\item le vote à la majorité de la totalité de ses membres lors d'un collège ;
	\item la perte de qualité de membre de l'association.

\end{enumerate}

Lorsque le nombre de membres de \troupe{} est strictement inférieur à six (6), les membres du \bureau{} participent aux décisions d'admission et d'exclusion de \troupe{}, ils doivent alors être présents aux collèges où sont prises ces décisions.

\section{Organisation des activités artistiques}
\article{Entraînements}
\label{sec:entrainements}
L'association organise deux types d'entraînements : les entraînements réguliers et les entraînements spectacles.

\subsubsection*{Entraînements réguliers}
L'association organise au moins une fois par mois un entraînement. Un animateur d'entraînement désigné par le \DA{} sera responsable du contenu de l'entraînement. Tous les membres adhérents et les membres d'honneur peuvent y participer. Tous les membres bienfaiteurs peuvent y assister. 

\subsubsection*{Entraînements spectacles}
Sur décision du \DA{}, des entraînements réservés à \troupe{} peuvent être organisés. Des membres adhérents ou membres d'honneur peuvent être invités par \troupe{} à y participer. Des membres peuvent être invités à y assister.

\article{Spectacles}
\label{sec:spectacles}
Le \DA{} est responsable de l'organisation des spectacles, entre autres :
\begin{itemize}
\item organiser les entraînements afin de créer un spectacle conforme au \PA{}.
\item trouver une salle capable d'accueillir le spectacle
\item assurer la communication et la publicité de l'évènement.
\end{itemize}

Le trésorier assistera le \DA{} et participera aux décisions quand au financement du spectacle. Le trésorier s'occupera également de la gestion des recettes.

Le \DA{} a toute autorité pour choisir à chaque représentation qui jouera le spectacle, en choisissant autant que possible des membres de \troupe{}, puis des membres adhérents ou des membres d'honneur.

Le \DA{} peut déléguer librement ses tâches.

\article{Licence des formats de spectacles}
\label{sec:licence-des-formats}
Conformément à l'\artrefst{sec:depot-et-diffusion}, les formats créés par l'association seront déposés sous licence Creative Commons (CC BY-SA).


\section{Disposition diverses}
\article{Règlement intérieur}
\label{sec:reglement-interieur}
Conformément à l'\artrefst{sec:reglement} de l'association, le \RI{} est établi par le \bureau{}.

Il peut être modifié sur demande de l'\AG{} ou du \bureau{}, suivant la procédure suivante :
\begin{itemize}
\item le \bureau{} rédige une proposition soumise à l'approbation de l'\AG{} ou par vote par courrier dans un délai de quinze (15) jours ;
\item le nouveau \RI{} est alors adressé à tous les membres par courrier.
\end{itemize}
\end{document}
