\documentclass[a4paper,french,10pt]{article}
\usepackage[french]{babel} 
\usepackage[utf8]{inputenc}
\usepackage[T1]{fontenc} %\usepackage{isolatin1} \usepackage{t1enc}
\usepackage{pslatex} % Pour une meilleure police avec le PDF
\usepackage{ae,aecompl,aeguill} 
\usepackage{marvosym} %pour l'¤
\usepackage{url} 
\usepackage{color} 
\usepackage{shadow}
\usepackage{fancyhdr} 
\usepackage{fancybox}
\usepackage[colorlinks=true,urlcolor=black,linkcolor=black,pdfauthor={Association
  META},pdftitle={Statuts de l'assocation
  META},pdfsubject={Association META : Improvisation Théâtrale, Basée
  à Paris},pdfkeywords={META, improvisation, impro, théâtre,
  spectacle, paris, association loi 1901, statuts }]{hyperref}
\usepackage{fancyheadings}
\pagestyle{fancyplain}


%Macros pour les noms
\newcommand{\meta}{\textbf{META}}
\newcommand{\metae}{\meta{} \textbf{Est une Troupe Associative}}
\newcommand{\AG}{Assemblée Générale}
\newcommand{\AGs}{Assemblées Générales}
\newcommand{\AGO}{\AG{} ordinaire}
\newcommand{\AGE}{\AG{} extraordinaire}
\newcommand{\AGsO}{\AGs{} ordinaires}
\newcommand{\AGsE}{\AGs{} extraordinaires}
\newcommand{\bureau}{bureau}
\newcommand{\RI}{règlement intérieur}
\newcommand{\troupe}{la Troupe}
\newcommand{\statuts}{statuts}
\newcommand{\DA}{directeur artistique}
\newcommand{\PA}{projet artistique}


\renewcommand{\headrulewidth}{0.4pt} %Trait haut
\renewcommand{\footrulewidth}{0pt} %Trait bas
\newcommand{\article}[1]{\subsection{#1}\addtocounter{article}{1}}
\newcounter{article}
\def\thesection       {\Roman{section}}
\def\thesubsection       {\arabic{article}}

\makeatletter
\def\@seccntformat#1{\@ifundefined{#1@cntformat}%
{\csname the#1\endcsname\quad}% default
{\csname #1@cntformat\endcsname}% individual control
}
\def\section@cntformat{Titre \thesection\quad}
\def\subsection@cntformat{Article \thesubsection\quad}
\makeatother
\newcommand{\artref}[1]{article \ref{#1}}
\title{Statuts de l'association \metae{}}





\begin{document}

\date{}

\maketitle 
\pagestyle{fancy} 

\lhead{\textit{Statuts de l'association \meta{}}}

%Haut gauche \chead{} 
%Haut centre \rhead{\thepage} 
%Haut droite
\rhead{\thepage} 
%Bas gauche 
\cfoot{} 
%Bas centre \rfoot{} 
%Bas droite \lfoot{}


\setcounter{article}{1}
\section{Présentation de l’association}
\article{Constitution et dénomination}
\label{sec:constitution}
Il est fondé entre les adhérents aux présents \statuts{} une association
régie par la Loi 1901, ayant pour dénomination \textit{\metae{}}, ci-après désignée par \textit{\meta{}} ou par \textit{l'association}.

\article{But}
\label{sec:but}

Cette association a pour but :
\begin{itemize}
\item la pratique, la promotion et la diffusion de l'improvisation théâtrale au travers d'entraînements et de spectacles ;
\item la création et la diffusion de formats de spectacles d'improvisation originaux ;
\item la réflexion autour de l'improvisation.
\end{itemize}

\article{Siège social}
\label{sec:siege}

Le siège social est fixé à l'adresse du président de l'association. Il
pourra être transféré par simple décision de la collectivité des
membres.

\article{Moyens d’action}
\label{sec:moyens}

Les moyens d’action de l’association sont notamment:
\begin{itemize}
\item les publications, les cours, les conférences, les réunions de
travail ;
\item l’organisation de manifestations, spectacles et toute initiative
pouvant aider à la réalisation de l’objet de l’association ;
\item la vente permanente ou occasionnelle de tous produits ou
services entrant dans le cadre de son objet ou susceptible de
contribuer à sa réalisation.
\end{itemize}

\article{Durée de l’association}
\label{sec:duree}
La durée de l’association est illimitée.

\section{Composition de l’association}

\article{Composition de l’association}
\label{sec:composition}
L’association se compose de membres d'honneur, de membres bienfaiteurs
et de membres adhérents.

Les membres d'honneur sont désignés par l'\AG{} pour les
services qu'ils ont rendus ou rendent à l'association. Ils sont
dispensés du paiement de la cotisation annuelle et ont le droit de
participer à l'\AG{} avec voix délibérative.

Les membres bienfaiteurs qui acquittent une cotisation spéciale fixée
par le \RI{} ont le droit de participer à l'\AG{} avec voix délibérative.

Les membres adhérents personnes physiques ou morales acquittent une
cotisation fixée annuellement par le \RI{}. Ils sont
membres de l'\AG{} avec voix délibérative.

\article{Admission et adhésion}
\label{sec:admission}

Pour faire partie de l’association, il faut être agréé par le
\bureau{} qui statue, selon les modalités prévues par le \RI{}, sur
les demandes d'admission présentées. Il faut alors adhérer aux
présents \statuts{} et au \RI{} et s’acquitter de la cotisation dont
le montant est fixé dans le \RI{}.

\article{Perte de la qualité de membre}
\label{sec:perte}
La qualité de membre se perd par :
\begin{itemize}
\item la démission adressée par écrit à un membre du \bureau{} de l'association,

\item le décès,

\item l'exclusion ou radiation, prononcées par le \bureau{} pour infraction
aux \statuts{} ou au \RI{}, ou pour motif portant préjudice
aux intérêts moraux et matériels de l'association, ou pour motif
grave.
\end{itemize}


\article{Responsabilité des membres}
\label{sec:responsabilite}
Aucun des membres de l’association n’est personnellement responsable
des engagements contractés par elle. Seul le patrimoine de
l’association répond de ses engagements.


\section{Organisation et fonctionnement}

\article{Moyen de correspondance}
\label{sec:moyen-de-corresp}
Les adhérents peuvent adresser leur correspondance écrite par lettre signée. Alternativement, l'adhérent peut choisir un moyen de communication sécurisé précisé dans le \RI{}. Les moyens précisés dans le \RI{} devront être conformes à la loi 2000-230 du 13 mars 2000 portant adaptation du droit de la preuve aux technologies de l'information et relative à la signature numérique, ils devront suivre les dispositions décrites dans le décret d'application 2001-272 du 30 mars 2001. 

\article{Décisions collectives des membres}
\label{sec:decisions-collectives}

Les décisions collectives des membres sont prises, soit en \AG{}, soit
par voie de consultation écrite. Elles peuvent encore résulter du
consentement de tous les membres exprimé dans un acte authentique ou
sous seing privé. Sous réserve des dispositions de l’\artref{sec:ago},
tout membre de l’association peut soumettre à la collectivité un
projet de décision collective.

En cas de consultation écrite, le \bureau{} envoie à chaque membre le
texte des résolutions proposées accompagné des documents nécessaires à
l’information des membres. Les membres disposent d’un délai de quinze
jours francs à compter de la date de réception des textes des
résolutions pour émettre leur vote par écrit. Le vote est formulé sur
le texte même des résolutions proposées et pour chaque résolution, par
le mot : “oui”, “non” ou “abstention”.

La conclusion d’un emprunt bancaire ou d’un contrat de travail par
l’association ne peut résulter que d’une décision collective des
membres.

\article{Assemblée Générale Ordinaire}
\label{sec:ago}
L’\AGO{} se réunit au moins une fois par an et comprend tous les membres de l’association à jour de leur cotisation.

Quinze jours au moins avant la date fixée, les membres de
l’association sont convoqués par écrit et l’ordre du jour est inscrit
sur les convocations.

%O: c'est un peu chiant par écrit, nan ?

Le \bureau{} expose la situation morale ou le rapport d'activité de
l'association puis rend compte de l'exercice financier. Le bilan est
soumis à l'approbation de l'assemblée. 
L’\AG{} se prononce sur le rapport moral ou d’activité et sur les comptes de l’exercice financier. Elle délibère sur les orientations à venir.

Elle pourvoit à la nomination ou au renouvellement des membres du \bureau{}, suivant la procédure fixée par le \RI{}. Elle fixe aussi le montant de la cotisation annuelle, autorise les emprunts bancaires et la conclusion de contrats de travail.


\article{Assemblée Générale Extraordinaire}
\label{sec:age}

Sur demande d'au moins la moitié des membres inscrits ou de sa propre initiative, le \bureau{} peut convoquer une \AGE{}.

\article{Modalités communes aux \AG{} ordinaire et extraordinaires}
\label{sec:ag}
L'ordre du jour est établi par le \bureau{}. % ou le président ?
% Seuls les points à l'ordre du jour peuvent être votés

Les membres peuvent soumettre par écrit au \bureau{} des points à ajouter à l'ordre du jour jusqu'à 7 jours avant la date de l'\AG{}.


Aucun quorum n'est requis. Le vote est effectué à bulletin secret ou à
main levée, suivant la nature de la décision : le choix est laissé à
l'initiative du président, excepté l'élection des membres du \bureau{}
pour laquelle le scrutin secret est requis.
% enlever le "excepté [...]" ?
% O : Non je pense pas

Les décisions de l’assemblée sont prises à la majorité relative des
membres présents et représentés. En cas de litige, le vote du
président prévaut.

Seuls les membres de l’association à jour de leur cotisation
bénéficient du droit de vote et sont éligibles.
% Ajout possible :
% Seuls les membres inscrits au moins deux mois avant la tenue de
% l'assamblée bénéficient du droit de vote.

% O: Bof, à la limite, dans le RI s'il faut.




\article{Bureau}
\label{sec:bureau}
L’\AG{} désigne, parmi les membres, au scrutin secret, un
\bureau{} composé de :
\begin{itemize}
\item un(e) président(e),
  
\item un(e) trésorier(e),

\item un(e) secrétaire.
\end{itemize}

Le \bureau{} prépare les réunions des membres. Il exécute les décisions de l’assemblée et traite les affaires courantes de l’association.

\article{Pouvoir du \bureau{}}
\label{sec:pouvoirs}
Le \bureau{} est investi des pouvoirs les plus étendus dans les limites
de l’objet de l’association et dans le cadre des résolutions adoptées
par décision collective des membres. Il peut autoriser tous actes ou
opérations qui ne sont pas statutairement de la compétence de
l’\AG{}.

Il est chargé :
\begin{itemize}
\item de la mise en \oe{}uvre des orientations décidées par l'\AG{},

  \item de la préparation des bilans, de l'ordre du jour et des
  propositions de modification du \RI{} présentés à
  l'\AG{},

  \item de la préparation des propositions de modifications des \statuts{}
  présentés à l'\AGE{}.

  \item de la préparation du \RI{}.

  
\end{itemize}



\article{Rémunération}
\label{sec:remuneration}
Les fonctions de membres du \bureau{} sont bénévoles ; seuls les frais et
débours occasionnés pour l’accomplissement du mandat social sont
remboursés au vu des pièces justificatives.

\article{La Troupe}
\label{sec:troupe}
Un collège de membres appelé \textit{\troupe{}} est chargé de
l'organisation des spectacles de l'association. Les représentations de
ces spectacles sont données par \troupe{}. Ce collège de membre est
dirigé par un membre nommé \DA{}. La désignation du \DA{} ainsi que le
fonctionnement de ce collège est fixé par le \RI{}.

\article{Décisions extraordinaires}
\label{sec:decisions-extra}
Les décisions collectives des membres portant sur la dissolution de
l’association, la modification des \statuts{}, la conclusion d’un emprunt
bancaire ou d’un contrat de travail par l’association doivent être prises en \AG{} et requièrent l’unanimité des votants.

\article{Règlement intérieur}
\label{sec:reglement}
Un \RI{} est établi par le \bureau{} qui le fait approuver
par l’\AG{}.

Il régit entre autre le montant des cotisations des adhérents, les règles d'exclusions et les modalités de fonctionnement de \troupe{}, ainsi que l'organisation du fonctionnement de l'association. En cas de contradiction entre les \statuts{} et le \RI{}, les \statuts{} prévalent.

% O:Formulation à revoir/compléter

\section{Orientation artistique et réflexion autour de l'improvisation}
\article{Organisation de réflexions}
\label{sec:reflexions}
L'association organisera au moins une fois par an une réflexion autour
de l'improvisation (conférence, table ronde, \AG{}, ou
tout autre moyen que l'association jugera bon d'employer).


\article{Projet artistique}
\label{sec:projet}
Suite à la réflexion menée suivant les conditions de l'\artref{sec:reflexions}, le \bureau{} définira les orientations artistiques pour \troupe{}. Il confiera l'établissement et la réalisation d'un \PA{} suivant ces orientations au \DA{} de \troupe{}. Ce projet devra comprendre au moins un format de spectacle d'improvisation innovant et original.

% Reply Edit May 30 Cyril:
% Donc, si je comprends bien, cela revient à dire que l'on doit produire
% au moins un format original par an ? C'est pas un peu dangereux de
% mettre ça dans les statuts ? Non je veux dire : "un chercheur s'engage
% à trouver la solution à au moins un problème ouvert par an", est une
% jolie illustration de mon scepticisme.  

% Reply Edit May 30 Ouardane:
% Hum... Il faudrait que je les rédige mieux, mais je pense que la
% modification ou la personnalisation d'un format déjà existant peut
% convenir. Après, vu le nombre d'idées qu'on a eu : 7to smoke, fantôme,
% les unités, le format créé par les spectateurs, etc... On a largement
% de quoi tenir. Et puis je fais confiance à notre créativité. Et on
% peut toujours trouver, même si c'est pas top, ça nous force pas à le
% produire trop souvent...  

% Reply Edit May 30 Drébon:
% En effet, je pense qu'expérimenter en public un format innovant par
% an, c'est largement jouable. Ça veut pas dire faire un truc chiadé et
% aboutit, mais le but ici, c'est justement faire de la recherche en
% improvisation. Ça donne un peu plus de raison d'être à l'association
% que simplement faire une troupe d'impro. Ça apporte une contribution
% originale au monde de l'improvisation et ça crée un outil de
% réflexion.  Ce projet sera soumis à l'approbation de l'AG. En cas de
% rejet de l'AG, le bureau rédigera un nouveau projet en tenant compte
% de la discussions lors de l'AG.


\article{Dépôt et diffusion des formats}
\label{sec:depot-et-diffusion}
Les nouveaux formats créés dans le \PA{} (voir \artref{sec:reflexions}) seront déposés sous licence libre précisé par le \RI{}. Ils seront diffusés sur Internet et tout autre moyen jugé approprié par l'\AG{}.

% Par l'AG ou un organe plus réactif (Bureau ?)
% Reply Edit May 22 Cyril:
% CC BY-SA ? BY-SA-NC ?

% Reply Edit May 22 Drébon:
% A priori je serai plutôt SA que NC...

% Reply Edit May 22 Ouardane:
% Pareil, créative commons semble être la plus facile à mettre en place
% et la plus adaptée. Après faudrait fouiller les détails de
% "commercial". Si c'est pour faire entrer de l'argent dans une
% association sans profits, c'est peut être pas considéré comme
% commercial, on pourrait donc pousser jusqu'à NC.  

% Reply Edit May 22 Drébon:
% Ben une des questions à se poser, c'est "veut-on autoriser des gens à
% utiliser nos concepts pour gagner de l'argent ?"
% De mon point de vue, ce n'est pas dérangeant... Du coup pourquoi pas
% un CC-BY-SA...


\section{Ressources de l’association}

\article{Ressources de l’association}
\label{sec:ressources}
Les ressources de l’association se composent :
\begin{itemize}
\item des cotisations ;
  
\item des subventions de l’État, des collectivités territoriales et
des établissements publics ;

\item du produit des manifestations qu’elle organise ;

\item des intérêts et redevances des biens et valeurs qu’elle peut
posséder ;

\item des rétributions des services rendus ou des prestations fournies
par l'association ;

\item de dons manuels ;

\item de toutes autres ressources autorisées par la loi, notamment, le
recours en cas de nécessité, à un ou plusieurs emprunts bancaires ou
privés.
\end{itemize}


\section{Dissolution de l’association}

\article{Dissolution}
\label{sec:dissolution}
La dissolution est prononcée par l'\AGO{} ou \AGE{}. Conformément à \artref{sec:decisions-collectives} elle doit être approuvée à l'unanimité des membres votants à l'\AG{}, de plus la présence de la totalité du \bureau{} est nécessaire.

En cas de dissolution, l’\AG{} désigne un ou plusieurs
liquidateurs qui seront chargés de la liquidation des biens de
l’association et dont elle détermine les pouvoirs.

Les membres de l’association ne peuvent se voir attribuer, en dehors
de la reprise de leurs apports financiers, mobiliers ou immobiliers,
une part quelconque des biens de l’association.

L’actif net subsistant sera attribué obligatoirement à une ou
plusieurs associations poursuivant des buts similaires et qui seront
désignés par l’\AG{}.


\end{document}
