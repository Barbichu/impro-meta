\section{Compte-rendu de l'AG fondatrice}
Présentation du bureau issu de cette AG, présentation des statuts et du \RI{}.
Notification du prêt contracté par l'association au près de trois personnes : Cyril Cohen, Ouardane Jouannot et Andréa Loyer.
\section{Admission des nouveaux membres}
Le bureau procèdera à l'admission des nouveaux membres. Ils faudra que ces derniers adhèrent aux statuts ainsi qu'au \RI{}.
\section{Création de \troupe{}}
Conformément à l'article\ref{ri-sec:fonctionnement-troupe}, la création de \troupe{} sera effectuée.
\section{Définition des directives artistiques}
Le bureau présentera les orientations artistiques qu'il compte donner à \troupe{}. Suite à cet exposé, la \troupe{} fera une proposition de \DA{}.
\section{Achats de matériel}

Du fait du format « série » que \meta{} donne au Papille, il faut filmer les spectacles. Pour le moment, Yohan utilise son matériel et en fait l'installation. Il serait probablement plus sain que l'association dispose de sa propre caméra.

Tenue de scène. Il pourrait être intéressant de définir une tenue de scène facilement identifiable pour les spectacles. Des T-shirts avec un floquage avaient été évoqués.

Dans un avenir plus lointain, on peut aussi envisager l'achat de projecteurs.

Sont donc soumis au vote de l'AG l'accord de principe et un budget maximal pour chacun de ces postes ainsi que leur définition précise.

\begin{myitemize}
	\item Caméra,
	\item Tenues de scène,
	\item Projecteurs.
\end{myitemize}

\section{Questions diverses}

